\documentclass[a4paper,12pt]{article}

\usepackage[utf8]{inputenc}
\usepackage{amsmath}
\usepackage{amssymb}
\usepackage{graphicx}
\usepackage{hyperref}
\usepackage{geometry}
\usepackage{float}
\usepackage{caption}
\usepackage{subcaption}
\usepackage{enumitem} 
\usepackage{booktabs}
\usepackage{xcolor}


\geometry{margin=1in}

\title{Project 1: Scam detection with Naive Bayes}
\author{Kevin Yu \\ Student ID: 1462539 \\ The University of Melbourne}
\date{\today}

\begin{document}

\maketitle

\newpage

% Sections
\section{Supervised model training}

The prior probability of \textbf{non-malicious class was $0.7995$} (rounded to $4$ d.p.), while the \textbf{scam class was $0.2005$}. This indicates that the dataset is imbalanced, with a higher proportion of non-malicious instances. The distribution aligns well with real-world data, where non-malicious messages significantly outnumber spam. \\

Table \ref{tab:probable_words} shows the top 10 most ``probable'' words for each class, based on the conditional probability $P(w \mid c)$. In contract, Table \ref{tab:predictive_words} shows the top 10 most ``predictive'' words for each class, calculated by the ratio, $P(w \mid c) / P(w \mid \neg c)$. Notably, the words in each table differ; although both measure the association of words with a particular class, they serve a slightly different purpose. 

For example, most probable words include words such as `call' ($p = 0.0274$), `customer' ($p = 0.0092$) and `reply' ($p = 0.0080$). These words appear frequently in scam messages, which makes them probable; however, intuitively, we expect legitimate messages to frequently contain such words, especially those from customer service. In contrast, the most predictive words include terms such as `prize' ($p = 88.80$) and `claim' ($p = 41.21$). These words can appear in legitimate context, however, they are far more likely to be used in scam messages. (How often do you randomly win a prize?) 

For non-malicious messages, the most predictive words include slang such as `lor' ($p = 32.16$), `wat' ($p = 19.53$) and `lol' ($p = 17.80$). These casual words are more likely to be used in non-malicious messages, and are not commonly found in scam messages. This serves as a good example of how the model can learn to distinguish between the two classes based on word usage.

Overall, the result is reasonable and it seems feasible to distinguish between scam and non-malicious messages using a multinomial naïve Bayes model. The differences in both frequent and predictive vocabulary across classes suggest the model can effectively learn class-specific language patterns, despite the simplicity of its assumptions.


\begin{table}[h!]
    \centering
    \begin{tabular}{|l|c||l|c|}
    \hline
    \multicolumn{2}{|c||}{\textbf{Scam Class}} & \multicolumn{2}{c|}{\textbf{Non-malicious Class}} \\
    \hline
    \textbf{Word} & \textbf{Probability} & \textbf{Word} & \textbf{Probability} \\
    \hline
    call      & 0.0274 & go     & 0.0161 \\
    free      & 0.0137 & get    & 0.0143 \\
    claim     & 0.0100 & gt     & 0.0085 \\
    customer  & 0.0092 & lt     & 0.0084 \\
    txt       & 0.0090 & call   & 0.0083 \\
    ur        & 0.0085 & ok     & 0.0078 \\
    text      & 0.0082 & come   & 0.0075 \\
    stop      & 0.0082 & ur     & 0.0075 \\
    reply     & 0.0080 & know   & 0.0075 \\
    mobile    & 0.0078 & good   & 0.0071 \\
    \hline
    \end{tabular}
    \caption{Top 10 most probable words for each class}
    \label{tab:probable_words}
\end{table}
    
\begin{table}[h!]
    \centering
    \begin{tabular}{|l|c||l|c|}
    \hline
    \multicolumn{2}{|c||}{\textbf{Scam Class}} & \multicolumn{2}{c|}{\textbf{Non-malicious Class}} \\
    \hline
    \textbf{Word} & \textbf{$P(w \mid \text{scam}) / P(w \mid \text{ham})$} & \textbf{Word} & \textbf{$P(w \mid \text{ham}) / P(w \mid \text{scam})$} \\
    \hline
    prize   & 88.80 & gt    & 60.30 \\
    tone    & 57.46 & lt    & 59.73 \\
    select  & 41.79 & lor   & 32.16 \\
    claim   & 41.21 & hope  & 27.57 \\
    50      & 34.82 & ok    & 27.57 \\
    paytm   & 33.08 & da    & 22.40 \\
    code    & 31.34 & let   & 20.10 \\
    award   & 28.73 & wat   & 19.53 \\
    won     & 27.86 & oh    & 18.38 \\
    18      & 26.12 & lol   & 17.80 \\
    \hline
    \end{tabular}
    \caption{Top 10 most predictive words for each classe based on likelihood ratios}
    \label{tab:predictive_words}
\end{table}

\section{Methodology}
Describe the methods and techniques used in your work. Include equations, algorithms, or diagrams if necessary.

\subsection{Data Preprocessing}
Explain how the data was prepared for analysis, including any cleaning, transformation, or feature extraction steps.

\subsection{Model Description}
Provide details about the model or approach used, such as Naive Bayes, decision trees, etc.

\section{Results}
Present the results of your analysis or experiments. Use tables, figures, and graphs to support your findings.

\begin{table}[H]
    \centering
    \begin{tabular}{lcc}
        \toprule
        Metric & Value 1 & Value 2 \\
        \midrule
        Accuracy & 0.95 & 0.90 \\
        Precision & 0.92 & 0.88 \\
        Recall & 0.93 & 0.89 \\
        \bottomrule
    \end{tabular}
    \caption{Example table caption.}
    \label{tab:example}
\end{table}

\section{Discussion}
Interpret the results and discuss their implications. Highlight any limitations or challenges encountered.

\section{Conclusion}
Summarize the key findings and contributions of the report. Suggest future work or improvements.

\section*{References}
\begin{enumerate}
    \item Author Name, \textit{Title of the Paper/Book}, Publisher, Year.
    \item Author Name, "Title of the Article," \textit{Journal Name}, vol. X, no. Y, pp. Z, Year.
\end{enumerate}


Since the dataset is already preprocessed, I'm directly supplying vocabulary=vocabulary to CountVectorizer without calling fit(), to avoid any unintended token filtering (e.g., removing tokens like 'hi') that may occur with the default tokenizer during fitting.

\end{document}

